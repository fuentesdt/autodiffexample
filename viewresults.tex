\documentclass[10pt]{amsart}
\usepackage[left=.15in,right=.15in,top=.15in,bottom=.15in]{geometry}
\geometry{letterpaper}                   % ... or a4paper or a5paper or ... 
%\geometry{landscape}                % Activate for for rotated page geometry
%\usepackage[parfill]{parskip}    % Activate to begin paragraphs with an empty line rather than an indent
\usepackage{color,graphicx}
\usepackage{amssymb}
\usepackage{amsmath}
\usepackage{verbatim}
%\DeclareGraphicsRule{.tif}{png}{.png}{`convert #1 `dirname #1`/`basename #1 .tif`.png}
\newcommand{\eqn}[1]{(\ref{#1})}

%\renewcommand{\labelenumi}{\arabic{enumi}. }

%http://latex2rtf.sourceforge.net/latex2rtf_1_9_19.html#Conditional-Parsing
%Starting with LaTeX2RTF 1.9.18, there is a handy method for
%controlling which content should be processed by LaTeX or by
%LaTeX2RTF . Control is achieved using the standard \if facility of
%TeX. If you include the following line in the preamble of your document 
%
%     \newif\iflatextortf
%Then you will create a new \iflatextortf command in LaTeX . TeX
%sets the value of this to false by default. Now, LaTeX2RTF
%internally sets \iflatextortf to be true, and to ensure that this
%is always the case, LaTeX2RTF ignores the command
%\latextortffalse. This means that you can control how different
%applications process your document by
%
%     \iflatextortf
%     This code is processed only by latex2rtf
%     \else
%     This code is processed only by latex
%     \fi
%Note that \iflatextortf will only work within a section; you
%cannot use this command to conditionally parse code that crosses
%section boundaries. Also, it will only work on complete table or
%figure environments. Due to the mechanism used by LaTeX2RTF in
%processing these environments, at this time the only way to
%conditionally parse tables and figures is to include two complete
%versions of the environment in question, nested within an
%appropriate \iflatextortf structure.
%
\usepackage{boxedminipage,float}
\usepackage{wrapfig,setspace}
\usepackage[pdftex, plainpages=false, colorlinks=true, citecolor=black, filecolor=black, linkcolor=black, urlcolor=black]{hyperref}


\newcommand{\picdir}{./pdffig}
\newcommand\viewdata[1]{
\begin{figure}[H]
\begin{tabular}{cccc} 
 \IfFileExists{history#1.png}{\includegraphics[width=0.24\textwidth]{{history#1}.png}}{\fbox{\begin{picture}(10,10)\put(0,5){ not found } \end{picture}}}
&\IfFileExists{OptMxy#1.png}{\includegraphics[width=0.24\textwidth]{{OptMxy#1}.png} }{\fbox{\begin{picture}(10,10)\put(0,5){ not found } \end{picture}}}
&\IfFileExists{OptFA#1.png}{\includegraphics[width=0.24\textwidth]{{OptFA#1}.png}  }{\fbox{\begin{picture}(10,10)\put(0,5){ not found } \end{picture}}}
&\IfFileExists{OptMz#1.png}{\includegraphics[width=0.24\textwidth]{{OptMz#1}.png}  }{\fbox{\begin{picture}(10,10)\put(0,5){ not found } \end{picture}}}
\end{tabular}           
\vspace{-15pt}
\caption{NG=number gauss points. NU=number uncertain var. #1}\label{Fig:#1}
\end{figure}
}

\author{
        D.~Fuentes\textsuperscript{1} \
}

\date{ \small
The University of Texas M.D. Anderson Cancer Center,\\
Departments of \textsuperscript{1}Imaging Physics, \textsuperscript{2}Diagnostic Radiology,
\textsuperscript{3}Gastrointenstinal Oncology,
and \textsuperscript{4}Biostatistics, Houston TX 77030, USA \\
%Email: \texttt{jhazle@mdanderson.org}   \\
Received: date / Accepted: date
% Webpage: \texttt{http://wiki.ices.utexas.edu/dddas}
}





\begin{document}

\setlength{\unitlength}{.1cm}

%\begin{figure}[H]
%\begin{tabular}{cc} 
% \includegraphics[width=0.24\textwidth]{solversummaryNP1.png} & 
% \includegraphics[width=0.24\textwidth]{solversummaryNP3.png}
% \\
%(a) & (b)\\
%\end{tabular}           
%\caption{Inverse Solver Summary. (a) kpl inverse only (b)kpl, kve, t0 }\label{Fig:summary}
%\end{figure}

\viewdata{NG3Nu3adjSNR10}
\viewdata{NG4Nu3adjSNR10}
\viewdata{NG5Nu3adjSNR10}

%historyNG3Nu3constDirectSNR02.png  OptFANG3Nu3constDirectSNR08.png   OptMxyNG3Nu3constDirectSNR12.png  OptMzNG3Nu3constDirectSNR20.png
%historyNG3Nu3constDirectSNR05.png  OptFANG3Nu3constDirectSNR10.png   OptMxyNG3Nu3constDirectSNR15.png  OptMzNG3Nu3constDirectSNR22.png
%historyNG3Nu3constDirectSNR08.png  OptFANG3Nu3constDirectSNR12.png   OptMxyNG3Nu3constDirectSNR20.png  poptNG3Nu3constDirectSNR02.mat
%historyNG3Nu3constDirectSNR10.png  OptFANG3Nu3constDirectSNR15.png   OptMxyNG3Nu3constDirectSNR22.png  poptNG3Nu3constDirectSNR05.mat
%historyNG3Nu3constDirectSNR12.png  OptFANG3Nu3constDirectSNR20.png   OptMzNG3Nu3constDirectSNR02.png   poptNG3Nu3constDirectSNR08.mat
%historyNG3Nu3constDirectSNR15.png  OptFANG3Nu3constDirectSNR22.png   OptMzNG3Nu3constDirectSNR05.png   poptNG3Nu3constDirectSNR10.mat
%historyNG3Nu3constDirectSNR20.png  OptMxyNG3Nu3constDirectSNR02.png  OptMzNG3Nu3constDirectSNR08.png   poptNG3Nu3constDirectSNR12.mat
%historyNG3Nu3constDirectSNR22.png  OptMxyNG3Nu3constDirectSNR05.png  OptMzNG3Nu3constDirectSNR10.png   poptNG3Nu3constDirectSNR15.mat
%OptFANG3Nu3constDirectSNR02.png    OptMxyNG3Nu3constDirectSNR08.png  OptMzNG3Nu3constDirectSNR12.png   poptNG3Nu3constDirectSNR20.mat
%OptFANG3Nu3constDirectSNR05.png    OptMxyNG3Nu3constDirectSNR10.png  OptMzNG3Nu3constDirectSNR15.png   poptNG3Nu3constDirectSNR22.mat

\clearpage
\viewdata{NG5Nu3constDirectTotalSignalSNR02Hermite}
\viewdata{NG5Nu3constDirectTotalSignalSNR05Hermite}
\viewdata{NG5Nu3constDirectTotalSignalSNR10Hermite}
\viewdata{NG5Nu3constDirectTotalSignalSNR15Hermite}
\viewdata{NG5Nu3constDirectTotalSignalSNR20Hermite}

\clearpage
\viewdata{NG5Nu3constDirectSumQuadSNR02Hermite}
\viewdata{NG5Nu3constDirectSumQuadSNR05Hermite}
\viewdata{NG5Nu3constDirectSumQuadSNR10Hermite}
\viewdata{NG5Nu3constDirectSumQuadSNR15Hermite}
\viewdata{NG5Nu3constDirectSumQuadSNR20Hermite}

\clearpage
\viewdata{NG5Nu3interior-pointTotalSignalSNR02Hermite}
\viewdata{NG5Nu3interior-pointTotalSignalSNR05Hermite}
\viewdata{NG5Nu3interior-pointTotalSignalSNR10Hermite}
\viewdata{NG5Nu3interior-pointTotalSignalSNR15Hermite}
\viewdata{NG5Nu3interior-pointTotalSignalSNR20Hermite}

\clearpage
\viewdata{NG5Nu3interior-pointSumQuadSNR02Hermite}
\viewdata{NG5Nu3interior-pointSumQuadSNR05Hermite}
\viewdata{NG5Nu3interior-pointSumQuadSNR10Hermite}
\viewdata{NG5Nu3interior-pointSumQuadSNR15Hermite}
\viewdata{NG5Nu3interior-pointSumQuadSNR20Hermite}




\clearpage
\begin{figure}[H]
 \includegraphics[width=0.74\textwidth]{globalboxplot.png}
\caption{box plot}\label{Fig:boxsummary}
\end{figure}



\begin{figure}[h]
\begin{tabular}{ccc} 
 \includegraphics[width=0.3\textwidth]{solversummaryNP3control.png} & 
 \includegraphics[width=0.3\textwidth]{solversummaryNP3paretoP3L28Max.png} &
 \includegraphics[width=0.3\textwidth]{solversummaryNP3paretoP35L28Max.png}
\end{tabular}           
\caption{Solver Summary.}
\end{figure}


\begin{figure}[h]
\begin{tabular}{ccc} 
 \includegraphics[width=0.3\textwidth]{solversummaryNP3constMax.png} & 
 \includegraphics[width=0.3\textwidth]{solversummaryNP3interior-pointTotalSignalLB.png} &
 \includegraphics[width=0.3\textwidth]{solversummaryNP3interior-pointTotalSignalUB.png}
\end{tabular}           
\caption{Solver Summary.}
\end{figure}





\end{document}
